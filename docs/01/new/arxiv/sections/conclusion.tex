\section{Conclusion}

This paper introduced decision-valued maps as a diagnostic object for characterizing how discrete outcomes depend on representational choices. The object is a mapping from representations to decision identities, evaluated under a fixed snapshot and engine, and materialized through content-addressed identifiers and immutable artifacts.

DecisionDB implements this framework through a five-stage sweep protocol, a five-table relational schema, and a replay verification procedure. In a graph routing demonstration, one representation parameter preserves route identity across its tested range while another induces a discrete route change. Replay verification confirmed that all persisted decision identifiers are deterministically recoverable.

The framework is limited by its restriction to discrete outcomes, its reliance on fixed snapshots and engines, and the narrow empirical coverage reported here. Extending the approach to finer parameter grids, additional domains, and cross-snapshot comparison are directions for future work.

The contribution is infrastructural. Making representational dependence observable does not improve outcomes, but it is a prerequisite for understanding when discrete outcomes can be trusted and when they cannot.
