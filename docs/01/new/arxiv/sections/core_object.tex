\section{The Decision-Valued Map}

The central object of study is a mapping
\[
f \colon \mathcal{R} \to \mathcal{D},
\]
where $\mathcal{R}$ denotes a family of representations over a fixed snapshot $s$ and $\mathcal{D}$ denotes a set of discrete decision identities. For each representation $r \in \mathcal{R}$, the engine $E$ produces raw output $E(r)$, and the equivalence policy $\pi$ extracts a decision identity $d = \pi(E(r))$.

Three structural features of this map are observable through controlled variation of $\mathcal{R}$. Persistence regions are connected subsets of $\mathcal{R}$ over which $f$ is constant; within a persistence region, representational variation preserves the outcome. Boundaries are loci in $\mathcal{R}$ where $f$ changes value, separating two persistence regions with different decision identities. Fractures are boundaries where a small change in representation parameters induces a discrete identity change, indicating high sensitivity of the outcome to the representation.

The purpose of DecisionDB is to \emph{materialize} $f$, evaluating it at declared points in $\mathcal{R}$, storing the results as immutable artifacts, then making the resulting map queryable, replayable, auditable. Figure~\ref{fig:pipeline} illustrates this protocol.

\definecolor{erdHeader}{HTML}{e1d5e7}
\definecolor{erdStroke}{HTML}{b5739d}
\definecolor{arenaFill}{HTML}{f3eff7}

\begin{figure}[t]
  \centering
  \begin{tikzpicture}[
    box/.style={draw=erdStroke, rounded corners=4pt, align=center, inner sep=5pt, minimum height=13mm, font=\rmfamily\footnotesize, fill=erdHeader},
    smallbox/.style={draw=erdStroke, rounded corners=3pt, align=center, inner sep=4pt, minimum height=10mm, font=\rmfamily\scriptsize, fill=erdHeader!40},
    idbox/.style={draw=erdStroke, rounded corners=3pt, align=center, inner sep=3pt, minimum height=8mm, font=\rmfamily\scriptsize},
    arrow/.style={-{Latex[length=2.5mm, width=2mm]}, thick, erdStroke},
    eqlbl/.style={font=\rmfamily\scriptsize, erdStroke},
    arenalbl/.style={font=\rmfamily\scriptsize, text=black!65}
  ]

  % Arena enclosure
  \draw[erdStroke!40, rounded corners=6pt, fill=arenaFill] (-0.8, 1.0) rectangle (4.6, -0.7);
  \node[arenalbl, anchor=north west] at (-0.6, 0.9) {arena (invariant)};

  % Arena components inside the enclosure
  \node[box, text width=20mm] (snap) at (0.6, 0.2) {Snapshot $s$};
  \node[box, text width=20mm] (eng) at (3.4, 0.2) {Engine $E$};

  % Representation family (parallel, outside arena, feeding into engine)
  \node[smallbox, text width=22mm] (r1) at (7.0, 1.6) {$r_1$};
  \node[smallbox, text width=22mm] (r2) at (7.0, 0.2) {$r_2$};
  \node[smallbox, text width=22mm] (r3) at (7.0, -1.2) {$r_3$};
  \node[arenalbl, anchor=south] at (7.0, 2.2) {representation family};

  % Arrows from arena engine to each representation evaluation
  \draw[arrow] (eng.east) -- ++(0.4,0) |- (r1.west);
  \draw[arrow] (eng.east) -- (r2.west);
  \draw[arrow] (eng.east) -- ++(0.4,0) |- (r3.west);

  % Decision identifiers (output)
  \node[idbox, fill=blue!12] (d1) at (10.2, 1.6) {$d_A$};
  \node[idbox, fill=blue!12] (d2) at (10.2, 0.2) {$d_A$};
  \node[idbox, fill=orange!18] (d3) at (10.2, -1.2) {$d_B$};

  \draw[arrow] (r1.east) -- (d1.west);
  \draw[arrow] (r2.east) -- (d2.west);
  \draw[arrow] (r3.east) -- (d3.west);

  % Equivalence bracket for persistence region
  \draw[erdStroke!60, thick, decorate, decoration={brace, amplitude=4pt}] (10.8, 1.9) -- (10.8, -0.1);
  \node[arenalbl, anchor=west] at (11.1, 0.9) {persistence};

  % Boundary marker
  \draw[red!60, thick, dashed] (9.4, -0.5) -- (11.0, -0.5);
  \node[font=\rmfamily\scriptsize, text=red!50, anchor=west] at (11.1, -0.5) {boundary};

  \end{tikzpicture}
  \caption{Evaluation protocol for a decision-valued map. A single arena, defined by a data snapshot, a query and a computational engine, remains invariant throughout evaluation. A family of alternative representations of the same snapshot is varied within this arena. Each representation is passed through the same engine to produce a discrete decision identifier. Identical identifiers indicate persistence of decision identity across representational change, while a differing identifier marks a boundary where representational variation alone alters the outcome.}
  \label{fig:pipeline}
\end{figure}
