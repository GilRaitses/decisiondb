\section{The Decision-Valued Map}

The central object of study is a mapping
\[
f \colon \mathcal{R} \to \mathcal{D},
\]
where $\mathcal{R}$ denotes a family of representations over a fixed snapshot $s$ and $\mathcal{D}$ denotes a set of discrete decision identities. For each representation $r \in \mathcal{R}$, the engine $E$ produces raw output $E(r)$, and the equivalence policy $\pi$ extracts a decision identity $d = \pi(E(r))$.

Three structural features of this map are observable through controlled variation of $\mathcal{R}$. Persistence regions are connected subsets of $\mathcal{R}$ over which $f$ is constant; within a persistence region, representational variation preserves the outcome. Boundaries are loci in $\mathcal{R}$ where $f$ changes value, separating two persistence regions with different decision identities. Fractures are boundaries where a small change in representation parameters induces a discrete identity change, indicating high sensitivity of the outcome to the representation.

The purpose of DecisionDB is to \emph{materialize} $f$, evaluating it at declared points in $\mathcal{R}$, storing the results as immutable artifacts, then making the resulting map queryable, replayable, auditable. Figure~\ref{fig:pipeline} illustrates this pipeline.

\definecolor{erdHeader}{HTML}{e1d5e7}
\definecolor{erdStroke}{HTML}{b5739d}

\begin{figure}[t]
  \centering
  \begin{tikzpicture}[
    box/.style={draw=erdStroke, rounded corners=4pt, align=center, inner sep=6pt, minimum height=18mm, text width=27mm, font=\rmfamily, fill=erdHeader},
    arrow/.style={-{Latex[length=3.5mm, width=2.5mm]}, thick, erdStroke}
  ]
  \node[box] (snap) at (0,0) {Snapshot $s$\\[2pt]{\footnotesize\color{black!70} sealed inputs}};
  \node[box, text width=34mm] (rep) at (4.2,0) {Representation $r$\\[2pt]{\footnotesize\color{black!70} deterministic encoding}};
  \node[box] (eng) at (8.2,0) {Engine $E$\\[2pt]{\footnotesize\color{black!70} fixed procedure}};
  \node[box] (pol) at (12.0,0) {Policy $\pi$\\[2pt]{\footnotesize\color{black!70} decision identity $d$}};
  \node[box, text width=50mm] (map) at (6.0,-2.8) {Decision map table\\[2pt]{\footnotesize\color{black!70} links $s$, $r$, $E$, $d$, $\pi$ by content hash}};

  \draw[arrow] (snap.east) -- (rep.west);
  \draw[arrow] (rep.east) -- (eng.west);
  \draw[arrow] (eng.east) -- (pol.west);
  \draw[arrow] (pol.south) -- ++(0,-0.55) -| ([xshift=10mm]map.north);
  \draw[arrow] (snap.south) -- ++(0,-0.55) -| ([xshift=-10mm]map.north);
  \end{tikzpicture}
  \caption{Decision-valued map construction under a single arena. A data snapshot, computational engine and equivalence policy define an evaluation context that remains invariant. Within this context, multiple representations of the same snapshot are evaluated independently, producing discrete decision identities that are logged as immutable, content-addressed artifacts. The resulting map makes explicit which representational variations preserve decision identity (persistence regions) and where representational change alone induces a different outcome (boundaries), providing the substrate for replay verification and reuse admissibility analysis.}
  \label{fig:pipeline}
\end{figure}
