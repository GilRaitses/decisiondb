\section{Conclusion}

Decision-valued maps provide a diagnostic way to make the dependence of discrete outcomes on representational choices explicit. The object is a mapping from representations to decision identities, evaluated under a fixed snapshot and engine, then materialized through content-addressed identifiers and immutable artifacts.

DecisionDB implements this framework through a five-stage sweep protocol, a five-table relational schema, a replay verification procedure. In a graph routing demonstration, one representation parameter preserves route identity across its tested range while another induces a discrete route change. Replay verification confirmed that all persisted decision identifiers are deterministically recoverable.

The framework is limited by its restriction to discrete outcomes, its reliance on fixed snapshots and engines, the narrow empirical coverage reported here. A natural extension is to densify the parameter sweep, apply the same protocol across additional domains, introduce a cross-snapshot comparison procedure that treats ``same query, different snapshot'' as an explicit axis of reuse admissibility.

The contribution of this work is infrastructural, introducing a diagnostic layer that records how discrete decisions arise from families of representations. Decision-valued maps expose stability and fracture directly, allowing downstream reuse to be conditioned on explicit representational context rather than on confidence or performance summaries alone. Decision reuse becomes a compatibility question that can be checked mechanically before deployment across tasks and timescales.
