\section{Problem Scope}

We consider systems with the following structure:

\begin{enumerate}
\item A \textbf{snapshot} $s$: a frozen, immutable slice of external inputs over a declared time window. Any change to the world state produces a new snapshot.
\item A \textbf{representation} $r \in \mathcal{R}(s)$: a deterministic encoding of $s$, defined by explicit structural choices (kernels, thresholds, weighting rules, aggregation policies). Each representation is fully specified by a declared parameter set and generated by a versioned factory.
\item A \textbf{engine} $E$: a fixed computational procedure that consumes a representation and produces raw output. Engine configuration and version are held constant during analysis.
\item An \textbf{equivalence policy} $\pi$: a declared rule that reduces raw engine output to a discrete \textbf{decision identity} $d \in \mathcal{D}$. The policy defines when two raw outputs correspond to the same identity, independent of incidental numerical differences.
\end{enumerate}

The scope is diagnostic: we characterize when decision identity persists across representational variation and when it changes. We do not introduce training procedures, adaptive updates, gradient-based optimization, or online learning. Continuous outputs are in scope only when reduced to discrete identities via a declared policy.

This framing applies wherever discrete outcomes emerge from complex pipelines and representational choices may influence those outcomes. Examples include routing under alternative cost encodings, classification under alternative feature constructions, and resource allocation under alternative aggregation rules.
