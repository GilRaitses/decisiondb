\section{Problem Scope}

Decision-valued mapping addresses systems with the following structure. A \emph{snapshot} $s$ is a frozen, immutable slice of external inputs over a declared time window; any change to the world state produces a new snapshot. A \emph{representation} $r \in \mathcal{R}(s)$ is a deterministic encoding of $s$, defined by explicit structural choices such as kernels, thresholds, weighting rules and aggregation policies; each representation is fully specified by a declared parameter set and generated by a versioned factory. An \emph{engine} $E$ is a fixed computational procedure that consumes a representation and produces raw output, with engine configuration and version held constant during analysis. An \emph{equivalence policy} $\pi$ is a declared rule that reduces raw engine output to a discrete \emph{decision identity} $d \in \mathcal{D}$, defining when two raw outputs correspond to the same identity independent of incidental numerical differences.

The scope is diagnostic. The map characterizes when decision identity persists across representational variation and when it changes. It does not introduce training procedures, adaptive updates, gradient-based optimization, or online learning. Continuous outputs are in scope only when reduced to discrete identities via a declared policy.

This framing applies wherever discrete outcomes emerge from complex pipelines and representational choices may influence those outcomes. Examples include routing under alternative cost encodings, classification under alternative feature constructions, resource allocation under alternative aggregation rules.
