\section{Relation to Prior Infrastructure Work}

The perspective advanced here follows a recurring pattern in the development of technical infrastructure, in which latent dependencies within complex systems are rendered explicit, inspectable, and stable enough to support collective use. In software engineering, abstract data types formalized the separation between representation and observable behavior, allowing systems to evolve internally without collapsing external guarantees~\cite{liskov1978adt}. In database systems, write-ahead logging transformed durability and recovery from ad-hoc mechanisms into auditable, replayable state transitions~\cite{gray1993tp}. In empirical research, specification-curve analysis made visible the dependence of reported conclusions on analytic choices that were previously implicit~\cite{simonsohn2018spec}.

These precedents are not treated as peer systems or comparable products, but as examples of a shared infrastructural sensibility. In each case, the primary contribution was not a novel computational procedure or performance improvement, but the introduction of a diagnostic layer that made structural dependence observable and testable. Decision-valued mapping extends this tradition to settings where discrete outcomes emerge from complex analytical pipelines, and where representational choices exert nontrivial influence on outcome identity.
