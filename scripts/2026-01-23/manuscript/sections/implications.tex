\section{Discussion: What This Enables}

This section interprets the empirical structure of decision-valued maps produced by representational sweeps in terms of auditability, reproducibility, and system reliability. The emphasis is on diagnosing failure modes and structural sensitivity rather than improving outcomes or performance.

\subsection{Auditability via Explicit Decision Provenance}

Empirical Result Placeholder:
\begin{itemize}
  \item Experiment ID: \texttt{<EXP\_ID\_AUDIT>}
  \item Snapshot: \texttt{<SNAPSHOT\_ID>}
  \item Representation family: \texttt{<REP\_FAMILY>}
  \item Decision type: \texttt{<DECISION\_TYPE>}
\end{itemize}

This subsection will report how decision identities can be traced deterministically to representation specifications, engine versions, and equivalence policies. The empirical contribution will consist of a verified replay in which identical identifiers are recovered across independent executions, demonstrating end-to-end auditability.

\subsection{Reliability as Identity Persistence Under Variation}

Empirical Result Placeholder:
\begin{itemize}
  \item Experiment ID: \texttt{<EXP\_ID\_PERSISTENCE>}
  \item Representation parameters swept: \texttt{<PARAM\_LIST>}
\end{itemize}

This subsection will characterize regions of representation space over which decision identity remains unchanged. Persistence will be reported descriptively, without ranking or optimization, as a property of the materialized decision-valued map.

\subsection{Failure Precursors and Boundary Localization}

Empirical Result Placeholder:
\begin{itemize}
  \item Experiment ID: \texttt{<EXP\_ID\_BOUNDARY>}
  \item Observed boundary parameters: \texttt{<BOUNDARY\_PARAM\_RANGE>}
\end{itemize}

This subsection will document loci in representation space where decision identity changes. These boundaries will be interpreted as potential failure precursors that can be diagnosed prior to deployment or incident occurrence.
