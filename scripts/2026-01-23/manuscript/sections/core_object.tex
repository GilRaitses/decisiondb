\section{Core Object: Decision-Valued Maps}

The central object of study is a decision-valued map
\[
f : \mathcal{R} \rightarrow \mathcal{D},
\]
where $\mathcal{R}$ denotes a family of representations over a fixed snapshot and $\mathcal{D}$ denotes a set of discrete decision identities.

Each element $r \in \mathcal{R}$ is a fully specified, deterministic representation derived from the snapshot. The engine consumes $r$ and produces raw output, which is then reduced to a decision identity $d \in \mathcal{D}$ according to an equivalence policy. The equivalence policy defines when two raw outputs are considered to represent the same decision identity, independent of incidental numerical differences.

The purpose of the system is to make the mapping $f$ queryable, replayable, and auditable. By materializing this map across controlled variation in $\mathcal{R}$, it becomes possible to observe regions of identity persistence, identify boundary formation, and detect fractures where small representational changes induce discrete outcome changes.
