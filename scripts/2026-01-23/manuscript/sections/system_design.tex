\section{DecisionDB System Architecture}

DecisionDB implements a minimal diagnostic infrastructure for materializing and auditing decision-valued maps of the form
\[
f : \mathcal{R} \rightarrow \mathcal{D},
\]
where representations vary under a fixed snapshot and a fixed engine. The system is designed to make representational dependence observable without introducing new models, performance targets, or adaptive procedures.

\subsection{Identifiers and Content Addressing}

All core objects in DecisionDB are identified using content-addressed hashes computed over canonical JSON encodings. Canonicalization rules enforce deterministic serialization by sorting keys, preserving array order, excluding whitespace, serializing floats as strings, and including explicit version fields. Hashes are computed using SHA-256, and identifiers are formed by prefixing the first sixteen hexadecimal characters of the digest with a type-specific tag.

This scheme ensures that identical content always produces identical identifiers, enabling reproducible replay and audit across runs and environments.

\subsection{Core Entities}

DecisionDB tracks five primary entities.

\textbf{Snapshots} represent frozen slices of the world over a declared time window. A snapshot captures all external inputs required for downstream processing and is treated as immutable once created.

\textbf{Representations} are deterministic encodings of a snapshot. Each representation is generated by a declared factory and fully specified by its representation specification, namespace, and factory version. Representation parameters are explicitly separated from tuning or engine parameters.

\textbf{Engine runs} record executions of a fixed computational engine on a specific representation. Engine configuration and version are held constant across representational sweeps. Raw outputs are stored as immutable artifacts and linked by content hash.

\textbf{Decisions} are discrete outcome identities extracted from raw engine output using a declared equivalence policy. The equivalence policy defines when two outputs correspond to the same identity, independent of incidental numerical differences.

\textbf{Decision maps} materialize the mapping between representations, engine runs, and decision identities. This table constitutes the empirical object of analysis and supports queries over persistence, boundary formation, and fracture.
