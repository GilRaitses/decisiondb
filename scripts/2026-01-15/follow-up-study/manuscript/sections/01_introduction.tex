This follow-up study maintains the existing PAX pipeline while making aggregation choices explicit and
traceable. The current implementation computes camera-level stress from image-derived features and maps
those values to intersection-level stress for routing (\texttt{src/pax/stress/composite\_stress.py},
\texttt{src/pax/pathfinding/learned\_heuristic.py}). The refactor framing uses MAT761 Chapter 0 concepts
as grounding for how aggregation collapses should be evaluated, with explicit links in
\texttt{MAT761/docs/classnotes/20260113\_chapter0\_pax\_notes.tex} and
\texttt{MAT761/docs/officehours/Gil\_Raitses\_MAT761\_Aggregation\_Quotient\_Spaces\_Project\_Brief.tex}
\cite{mat761_chapter0_pax_notes,mat761_officehours_brief}.

The primary question is structural: which properties of the planning output are stable under different
aggregation constructions, and which are sensitive to bookkeeping choices such as distance cutoffs,
neighbor definitions, or street-name matching. The manuscript documents what is implemented today and
labels proposed comparisons without prescribing changes.

In a controlled perturbation of the Voronoi neighbor weight (0.5 \( \rightarrow \) 1.0), the route for the
fixed OD pair (Grand Central Terminal to Carnegie Hall, node 85 to node 50) did not change in node
sequence, edge ordering, distance, or cumulative stress in this pass. This empirical observation frames
the central research question as invariance of planning structure under representational perturbation,
motivating the use of MAT761 concepts like deformation and quotient-like aggregation without claiming
formal equivalence.

% course-substitution derivative text (2–3 sentences, reuse-ready)
% A single perturbation of the neighbor weight (0.5 to 1.0) left the chosen route unchanged, indicating
% that the decision structure can remain invariant under small representational changes. We treat this
% as a concrete illustration of stability and deformation in the MAT761 sense, using aggregation as the
% mapping whose continuity is under test.
