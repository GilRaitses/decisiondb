\begin{abstract}
This study examines how aggregation choices in the PAX routing pipeline affect downstream planning stability.
We compare intersection stress fields produced by street-based averaging, inverse-distance weighting, and
hard assignment, and track when route ordering, bottleneck locations, and low-stress connectivity change
under these representations. The goal is to identify which structural properties are invariant to
aggregation and which are sensitive to bookkeeping choices such as neighborhood cutoffs and weighting rules.
This framing is guided by MAT761 Chapter 0 notions of quotient constructions and continuity: the aggregation
map defines the representation, and stability is evaluated by whether small changes in the raw data alter
the induced planning structure. Results will be used to refine the refactor plan and to support a
teaching-facing, hands-off exploration interface for course-guided analysis.
\end{abstract}
