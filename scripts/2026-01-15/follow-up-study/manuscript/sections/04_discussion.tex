The current implementation contains discrete cutoffs and string-based routing context that introduce
brittleness in aggregation. IDW selection depends on k-nearest ordering and a max-distance cutoff
in \texttt{src/pax/stress/composite\_stress.py:compute\_intersection\_stress}. Street-based averaging
relies on street name matching in
\texttt{src/pax/pathfinding/learned\_heuristic.py:\_map\_intersections\_to\_stress}. These choices
are explicit in code and are the primary targets for stability comparisons in the proposed study.

The neighbor-weight perturbation (0.5 \( \rightarrow \) 1.0) in the Voronoi aggregation path did not change
the selected route for the fixed OD pair, and the edge ordering and cumulative stress remained unchanged.
This is a diagnostic outcome under the earlier zero-stress condition and should not be interpreted as
structural invariance once stress activation is used; any stability language must be conditional on the
stress field and the activation method.

Earlier invariance observations were invalid because stress coverage was zero in the mean\_stress
entrypoints. With activated normalized\_stress, the results now reflect nonzero stress penalties and
must be treated as a diagnostic correction rather than a discovery.

MAT761 notes emphasize quotient constructions and continuity via open-set preimages
(\texttt{MAT761/docs/classnotes/20260113\_chapter0\_pax\_notes.tex}) and treat aggregation as a mapping
problem (\texttt{MAT761/docs/officehours/Gil\_Raitses\_MAT761\_Aggregation\_Quotient\_Spaces\_Project\_Brief.tex}).
The manuscript will use these references to justify why structural stability (route ordering, bottlenecks)
is the appropriate evidence for refactor progress.

Under this configuration, Pearson's activation diagnosis confirms nonzero stress propagation
through zones, intersections, and edges with the OD 85 \(\rightarrow\) 50 run, enabling
perturbation outcomes to be interpreted as behavioral tests under the activated stress field
only. This scopes interpretation to the activation artifacts and deprecates the earlier
zero-stress interpretation as a diagnostic baseline. Evidence:
\texttt{pax/scripts/2026-01-16/artifacts/traces/pearson\_stress\_activation\_trace.json} and
\texttt{pax/scripts/2026-01-16/artifacts/audits/normalized\_stress\_audit.json}.
